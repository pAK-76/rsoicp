\chapter{Технологический раздел}
\label{cha:impl}

В данном разделе описываются технические средства,  используемые при проектировании распределенной системы обработки информации. Также приведены результаты разработки и тестирования системы. 

\section{Среда разработки, язык программирования}
Разработка распределенной системы осуществлялась на языке Java с использованием MVC библиотеки Play!
Язык Java был выбран, потому что он является переносимым и имеет обширную библиотеку стандартных функций, значительно ускоряющих разработку приложений. Фреймворк Play! был выбран, поскольку он содержит множество классов, помогающих в разработке Web-интерфейса участников и обмене сообщениями между ними. Среда IntelliJ IDEA была выбрана, так как является кроссплатформенной средой разработки, поддерживает выбранный фреймворк и бесплатна для использования в учебных целях.

\section{Выбор протоколов взаимодействия}

\subsection{Протокол асинхронного взаимодествия}

В качестве протокола асинхронного взаимодействия был выбран протокол SMTP, так как он является одним из рекомендуемых кафедрой протоколов для выполнения курсового проектирования.

\subsection{Протокол синхронного взаимодействия}

В качестве протокола синхронного взаимодействия был использован протокол HTTP/REST, так как он является наиболее удобным протоколом для реализации в MVC-фреймворке и позволяет унифицированно взаимодейтсвовать как пользователю с системой, так и системам между собой.

\section{Диаграммы классов}

% TODO
Тут будут всякие диаграммы классов

\section{Тестирование системы} 
Для проверки работоспособности системы будет (?) проведено тестирование как функционирования системы в целом, так и отдельных ее частей. 

Модульное тестирование будет(?_ проведено с использованием пакета Junit, который позволяет максимально просто написать модульные тесты для методов разработанных классов. Результаты тестирования приведены в приложении таком-то.%~\ref{cha:appendix1}.

%%% Local Variables:
%%% mode: latex
%%% TeX-master: "rpz"
%%% End:
